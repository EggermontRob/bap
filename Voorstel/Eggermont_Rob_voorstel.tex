%==============================================================================
% Sjabloon onderzoeksvoorstel bachelorproef
%==============================================================================
% Gebaseerd op LaTeX-sjabloon ‘Stylish Article’ (zie voorstel.cls)
% Auteur: Jens Buysse, Bert Van Vreckem
%
% Compileren in TeXstudio:
%
% - Zorg dat Biber de bibliografie compileert (en niet Biblatex)
%   Options > Configure > Build > Default Bibliography Tool: "txs:///biber"
% - F5 om te compileren en het resultaat te bekijken.
% - Als de bibliografie niet zichtbaar is, probeer dan F5 - F8 - F5
%   Met F8 compileer je de bibliografie apart.
%
% Als je JabRef gebruikt voor het bijhouden van de bibliografie, zorg dan
% dat je in ``biblatex''-modus opslaat: File > Switch to BibLaTeX mode.

\documentclass{voorstel}

\usepackage{lipsum}

%------------------------------------------------------------------------------
% Metadata over het voorstel
%------------------------------------------------------------------------------

%---------- Titel & auteur ----------------------------------------------------

% TODO: geef werktitel van je eigen voorstel op
\PaperTitle{Open source tools voor het opzetten, monitoren en updaten van containers binnen een Microsoft Azure omgeving: een vergelijkende studie.}
\PaperType{Onderzoeksvoorstel Bachelorproef 2019-2020} % Type document

% TODO: vul je eigen naam in als auteur, geef ook je emailadres mee!
\Authors{Eggermont Rob\textsuperscript{1}} % Authors
\CoPromotor{Joachim Dheedene\textsuperscript{2} (delaware)}
\affiliation{\textbf{Contact:}
  \textsuperscript{1} \href{mailto:rob.eggermont.y7223@student.hogent.be}{rob.eggermont.y7223@student.hogent.be};
  \textsuperscript{2} \href{mailto:Joachim.Dheedene@delaware.pro}{Joachim.Dheedene@delaware.pro};
}

%---------- Abstract ----------------------------------------------------------

\Abstract{
    Dit onderzoek gaat over de verschillende tools die gebruikt kunnen worden om containers te te publiceren, te monitoren en up te daten in een Microsoft Azure omgeving. Dankzij de overvloed aan beschikbare tools geraakt men snel het overzicht kwijt. Dit onderzoek gaat daarom op zoek naar een juiste set van tools om de uitrol, monitoring en wijziging van containers in Microsoft Azure een correcte workflow te geven om, op termijn, tijd en kosten te besparen.
}

%---------- Onderzoeksdomein en sleutelwoorden --------------------------------
% TODO: Sleutelwoorden:
%
% Het eerste sleutelwoord beschrijft het onderzoeksdomein. Je kan kiezen uit
% deze lijst:
%
% - Mobiele applicatieontwikkeling
% - Webapplicatieontwikkeling
% - Applicatieontwikkeling (andere)
% - Systeembeheer
% - Netwerkbeheer
% - Mainframe
% - E-business
% - Databanken en big data
% - Machineleertechnieken en kunstmatige intelligentie
% - Andere (specifieer)
%
% De andere sleutelwoorden zijn vrij te kiezen

\Keywords{Applicatieontwikkeling (andere). Monitoring --- CI/CD --- Microsoft Azure} % Keywords
\newcommand{\keywordname}{Sleutelwoorden} % Defines the keywords heading name

%---------- Titel, inhoud -----------------------------------------------------

\begin{document}

\flushbottom % Makes all text pages the same height
\maketitle % Print the title and abstract box
\tableofcontents % Print the contents section
\thispagestyle{empty} % Removes page numbering from the first page

%------------------------------------------------------------------------------
% Hoofdtekst
%------------------------------------------------------------------------------

% De hoofdtekst van het voorstel zit in een apart bestand, zodat het makkelijk
% kan opgenomen worden in de bijlagen van de bachelorproef zelf.
%---------- Inleiding ---------------------------------------------------------

\section{Introductie} % The \section*{} command stops section numbering
\label{sec:introductie}
Met oog op de snel veranderende en groeiende wereld van de IT wordt er steeds meer gebruikgemaakt van containers. Containers zorgen voor een snelle, flexibele en schaalbare oplossing voor verschillende  eisen die gesteld worden binnen de informatica wereld. Door de verschillende, kleine bouwstenen van deze infrastructuur gaat het overzicht echter snel verloren. Deze studie gaat op zoek naar open source tools die het opzetten, updaten en monitoren van containers in Microsoft Azure vereenvoudigt. Daarnaast bespreekt dit onderzoek ook tools om CI/CD toepassingen te integreren in de workflow die de uptime van de containers optimaal houden bij het uitrollen van een update.
\subsection{Onderzoeksvraag}
Welke open source tools kunnen worden gebruikt voor het efficiënt opzetten, monitoren en wijzigen van containers binnen Microsoft Azure?
\subsection{Onderliggende onderzoeksvragen}
\begin{itemize}
    \item Welke monitoring tools scoren het best op vlak van gebruiksvriendelijkheid?
    \item Welke monitoring tools rapporteren het snelst problemen die zich voordoen bij containers die reeds in gebruik zijn?
    \item Welke set van tools zorgen ervoor dat een container in productie zo snel mogelijk terug naar een werkende staat wordt gebracht na een fatale fout?
    \item Bij welke tools is een container binnen een Microsoft Azure omgeving het snelst beschikbaar?
    \item Bij gebruik van welke tools is er bij een update van een container op Microsoft Azure het minst (of geen) downtime?
    \item Hoe zorgen we ervoor dat containers op de gewenste manier blijven werken in productie omgevingen wanneer er zich fouten voordoen?
    \item Hoe kunnen we, door gebruik te maken van CI/CD, containers aanpassen/updaten in productie?
\end{itemize}
%---------- Stand van zaken ---------------------------------------------------

\section{Stand van zaken}
\label{sec:state-of-the-art}

Vandaag de dag worden er meer en meer (open srouce) tools beschikbaar gesteld voor het beheren en onderhouden van containers. Eén van de meest geanticipeerde tools is Kubernetes. Dit onderzoek gaat op zoek naar hulpprogramma's die een aanvulling kunnen bieden aan Kubernetes (K8's). Er zijn immens veel tools, bedoeld voor zowel het management als voor ontwikkelaars en systeembeheerders. Deze tools kunnen we onderverdelen in drie categorieën:

\subsection{CI/CD}
Continuous integration and continuous delivery ligt aan de basis voor ontwikkelaars. Het automatiseert het proces van code tot aflevering van de software (container in dit geval). Een eerste belangrijk onderdeel van CI/CD voor containers zijn de Package Managers. Deze worden geclassificeerd onder CD (continuous delivery) en staan in voor het 'inpakken' van de code naar een containerimage. Sommigen van deze hulpprogramma's staan zelfs in voor het uitrollen van de gemaakte image.\autocite{Anita,2018a}
Buiten de package management tools, zijn er verschillende andere tools die instaan voor continuous delivery zoals Weave Cloud, Spinnaker, Codefresh en Harness. Naast de tools voor continuous delivery zijn er die voor continuous integration. Alle developers werken in éénzelfde repository. Bij elke check-in wordt de ingevoerde code gecontrolleerd door een automatisch gestart proces. Dit proces behoort tot CI. Als alle taken slagen kan er aan deze tools gevraagd worden om de container image te maken en in de repository van de applicatie te plaatsen. Vele CI-tools hebben intussen ondersteuning toegevoegd om deze container images uit te rollen in Kubernetes clusters.
\subsection{Monitoring}
Om een goede werking van services te garanderen is een accurate monitoring belangrijker dan ooit. Monitoring software kan instaan voor rapportering naar de developer toe, maar ook naar andere software toe zodat de ontdekte fouten , waar mogelijk, geautomatiseerd opgelost kunnen worden.\autocite{Buehrle2018pt2}
Alle geschreven code moet uiteraard grondig getest worden vooraleer deze uitgerold wordt naar de eindgebruiker. Deze testen kunnen vaak automatisch verlopen. Om dit te laten gebeuren worden er hulpprogramma's gebruikt zoals:
\subsection{Security}
Containers die in een Kubernetes cluster zitten komen vaak terecht op een shared of een shared-private cloud omgeving terecht. Het is dus van groot belang het netwerk tussen de containers (of Pods in Kubernetes) te beveiligen. Dit wordt verwezenlijkt door een bepaald netwerkbeleid in te stellen op een cluster.
Er zijn duizenden tools die beschikbaar zijn. Het overzicht is dus eenvoudig te verliezen terwijl er duidelijk nood is aan een 'gouden combinatie'.

% Voor literatuurverwijzingen zijn er twee belangrijke commando's:
% \autocite{KEY} => (Auteur, jaartal) Gebruik dit als de naam van de auteur
%   geen onderdeel is van de zin.
% \textcite{KEY} => Auteur (jaartal)  Gebruik dit als de auteursnaam wel een
%   functie heeft in de zin (bv. ``Uit onderzoek door Doll & Hill (1954) bleek
%   ...'')
%---------- Methodologie ------------------------------------------------------
\section{Methodologie}
\label{sec:methodologie}

Er zal onderzoek gevoerd worden op basis van snelheid en gebruikte resources voor de 5 meest populaire tools per categorie. Dit op zowel het vlak van individuele werking als mogelijke samenwerking tussen de programma's onderling. De proof of concept zal bestaan uit een samenhang van tools die kan worden gebruikt van development tot monitoring en herstelling van containers in Kubernetes clusters. Door deze vergelijkende studie uit te voeren tussen de applicaties onderling zullen de voor- en nadelen van deze tools duidelijk worden. De applicaties zullen worden vergeleken op vlak van prestatie en onderlinge samenwerking (compatibiliteit). De belangrijkste parameter is de werking met Microsoft Azure AKS (Azure Kubernetes Service).


%---------- Verwachte resultaten ----------------------------------------------
\section{Verwachte resultaten}
\label{sec:verwachte_resultaten}

De verwachte resultaten van de proof of concept zullen de voordelen aantonen van de gekozen toolset voor het volledige ontwikkelproces van containers op Microsoft Azure binnen een Kubernetes cluster. Daarnaast wordt aangetoond dat de bekomen resultaten beter zijn dan die voor een andere kandidaat toolset. De resultaten zullen bestaan uit zowel de snelheid van de gekozen tools als de gebruikte resources. Als extra kunnen het aantal fouten die voorkomen gebruikt worden als doorslaggevende factor.


%---------- Verwachte conclusies ----------------------------------------------
\section{Verwachte conclusies}
\label{sec:verwachte_conclusies}

Na het uitvoeren van dit onderzoek verawchten we, bij benadering, een perfecre toolset voor het ontwikkelen en uitrollen van containers. Daarnaast breiden we deze toolset uit met programma's en bijhorende dashboard die contrainers monitoren op verschillende soorten problemen zoals te weninig geheugen, netwerkproblemen, ... 



%------------------------------------------------------------------------------
% Referentielijst
%------------------------------------------------------------------------------
% TODO: de gerefereerde werken moeten in BibTeX-bestand ``voorstel.bib''
% voorkomen. Gebruik JabRef om je bibliografie bij te houden en vergeet niet
% om compatibiliteit met Biber/BibLaTeX aan te zetten (File > Switch to
% BibLaTeX mode)

\phantomsection
\printbibliography[heading=bibintoc]

\end{document}
